%%%%%%%%%%%%%%%%%%%%%%%%%%%%%%%%%%%%%%%%%
% Compact Laboratory Book
% LaTeX Template
% Version 1.0 (4/6/12)
%
% This template has been downloaded from:
% http://www.LaTeXTemplates.com
%
% Original author:
% Joan Queralt Gil (http://phobos.xtec.cat/jqueralt) using the labbook class by
% Frank Kuster (http://www.ctan.org/tex-archive/macros/latex/contrib/labbook/)
%
% License:
% CC BY-NC-SA 3.0 (http://creativecommons.org/licenses/by-nc-sa/3.0/)
%
% Important note:
% This template requires the labbook.cls file to be in the same directory as the
% .tex file. The labbook.cls file provides the necessary structure to create the
% lab book.
%
% The \lipsum[#] commands throughout this template generate dummy text
% to fill the template out. These commands should all be removed when 
% writing lab book content.
%
% HOW TO USE THIS TEMPLATE 
% Each day in the lab consists of three main things:
%
% 1. LABDAY: The first thing to put is the \labday{} command with a date in 
% curly brackets, this will make a new section showing that you are working
% on a new day.
%
% 2. EXPERIMENT/SUBEXPERIMENT: Next you need to specify what 
% experiment(s) and subexperiment(s) you are working on with a 
% \experiment{} and \subexperiment{} commands with the experiment 
% shorthand in the curly brackets. The experiment shorthand is defined in the 
% 'DEFINITION OF EXPERIMENTS' section below, this means you can 
% say \experiment{pcr} and the actual text written to the PDF will be what 
% you set the 'pcr' experiment to be. If the experiment is a one off, you can 
% just write it in the bracket without creating a shorthand. Note: if you don't 
% want to have an experiment, just leave this out and it won't be printed.
%
% 3. CONTENT: Following the experiment is the content, i.e. what progress 
% you made on the experiment that day.
%
%%%%%%%%%%%%%%%%%%%%%%%%%%%%%%%%%%%%%%%%%

%----------------------------------------------------------------------------------------
%	PACKAGES AND OTHER DOCUMENT CONFIGURATIONS
%----------------------------------------------------------------------------------------                               

\documentclass[fontsize=11pt, % Document font size
paper=a4, % Document paper type
twoside, % Shifts odd pages to the left for easier reading when printed, can be changed to oneside
captions=tableheading,
index=totoc,
hyperref]{labbook}

\usepackage[bottom=10em]{geometry} % Reduces the whitespace at the bottom of the page so more text can fit

\usepackage[english]{babel} % English language
\usepackage{lipsum} % Used for inserting dummy 'Lorem ipsum' text into the template

\usepackage[utf8]{inputenc} % Uses the utf8 input encoding
\usepackage[T1]{fontenc} % Use 8-bit encoding that has 256 glyphs

\usepackage[osf]{mathpazo} % Palatino as the main font
\linespread{1.05}\selectfont % Palatino needs some extra spacing, here 5% extra
\usepackage[scaled=.88]{beramono} % Bera-Monospace
\usepackage[scaled=.86]{berasans} % Bera Sans-Serif

\usepackage{booktabs,array} % Packages for tables

\usepackage{amsmath} % For typesetting math
\usepackage{graphicx} % Required for including images
\usepackage{etoolbox}
\usepackage[norule]{footmisc} % Removes the horizontal rule from footnotes
\usepackage{lastpage} % Counts the number of pages of the document
\usepackage{shadowtext}
\usepackage{minted}
\usepackage[backend=biber,
style=authoryear-comp,
natbib=true,
]{biblatex}
\addbibresource{JNMansfield_CapstoneProduct.bib}

\usepackage{setspace}
\usepackage{url}
\usepackage[framemethod=tikz]{mdframed}

\addtokomafont{title}{\Huge\color{green!40!brown!80}} % Titles in custom blue color
\addtokomafont{chapter}{\color{green!40!brown!80}} % Lab dates in olive green
\addtokomafont{section}{\color{brown}} % Sections in sepia
\addtokomafont{pagehead}{\normalfont\sffamily\color{gray}} % Header text in gray and sans serif
\addtokomafont{caption}{\footnotesize\itshape} % Small italic font size for captions
\addtokomafont{captionlabel}{\upshape\bfseries} % Bold for caption labels
\addtokomafont{descriptionlabel}{\rmfamily}
\setcapwidth[r]{10cm} % Right align caption text
\setkomafont{footnote}{\sffamily} % Footnotes in sans serif

\deffootnote[4cm]{4cm}{1em}{\textsuperscript{\thefootnotemark}} % Indent footnotes to line up with text

\DeclareFixedFont{\textcap}{T1}{phv}{bx}{n}{1.5cm} % Font for main title: Helvetica 1.5 cm
\DeclareFixedFont{\textaut}{T1}{phv}{bx}{n}{0.8cm} % Font for author name: Helvetica 0.8 cm

\usepackage[nouppercase,headsepline]{scrpage2} % Provides headers and footers configuration
\pagestyle{scrheadings} % Print the headers and footers on all pages
\clearscrheadfoot % Clean old definitions if they exist

\automark[chapter]{chapter}
\ohead{\headmark} % Prints outer header

\setlength{\headheight}{25pt} % Makes the header take up a bit of extra space for aesthetics
\setheadsepline{.4pt} % Creates a thin rule under the header
\addtokomafont{headsepline}{\color{lightgray}} % Colors the rule under the header light gray

\ofoot[\normalfont\normalcolor{\thepage\ |\  \pageref{LastPage}}]{\normalfont\normalcolor{\thepage\ |\  \pageref{LastPage}}} % Creates an outer footer of: "current page | total pages"

% These lines make it so each new lab day directly follows the previous one i.e. does not start on a new page - comment them out to separate lab days on new pages
\makeatletter
\patchcmd{\addchap}{\if@openright\cleardoublepage\else\clearpage\fi}{\par}{}{}
\makeatother
\renewcommand*{\chapterpagestyle}{scrheadings}


% These lines make it so every figure and equation in the document is numbered consecutively rather than restarting at 1 for each lab day - comment them out to remove this behavior
\usepackage{chngcntr}
\counterwithout{figure}{labday}
\counterwithout{equation}{labday}

% Hyperlink configuration
\usepackage[
pdfauthor={}, % Your name for the author field in the PDF
pdftitle={Laboratory Journal}, % PDF title
pdfsubject={}, % PDF subject
bookmarksopen=true,
linktocpage=true,
pdfpagelabels=true,
plainpages=false,
colorlinks=true, % Turn off all coloring by changing this to false
bookmarks=true,
pdfview=FitB]{hyperref}

\usepackage[stretch=10]{microtype} % Slightly tweak font spacing for aesthetics

%\setlength\parindent{0pt} % Uncomment to remove all indentation from paragraphs

%----------------------------------------------------------------------------------------
%	DEFINITION OF EXPERIMENTS
%----------------------------------------------------------------------------------------

% Template: \newexperiment{<abbrev>}[<short form>]{<long form>}
% <abbrev> is the reference to use later in the .tex file in \experiment{}, the <short form> is only used in the table of contents and running title - it is optional, <long form> is what is printed in the lab book itself

\newexperiment{exp1}[The DataFace Application]{The DataFace Application}



%----------------------------------------------------------------------------------------

\begin{document}

%----------------------------------------------------------------------------------------
%	TITLE PAGE
%----------------------------------------------------------------------------------------

\title{\textcap{\shadowtext{CS493}\\
\textaut{\small{Regis University}}	
\textaut{\small{Senior Capstone}}\\		
\textaut{\shadowtext{Capstone Product Information}}
\textaut{\shadowtext{for Stakeholders}}}\\
}

\author{
\includegraphics[scale=0.5]{android.png}\\
\textaut{\small{Intructor: Mr. Allan Rossi}}\\
\textaut{\small{Author: Mr. Jason N Mansfield}}\\ \\ % Your name
The Object Based Study Application % Your degree
}
\date{\today} % No date by default, add \today if you wish to include the publication date

\maketitle % Title page

\printindex
\tableofcontents % Table of contents
\newpage % Start lab look on a new page


\pagestyle{scrheadings} % Begin using headers

%----------------------------------------------------------------------------------------
%	LAB BOOK CONTENTS
%----------------------------------------------------------------------------------------

\labday{\today}
%-----------------------------------------

\experiment{exp1}
\begin{onehalfspace}
This is a basic guide for retrieving the DataFace application in it current status. Please keep in mind the application is still very much in a beta stage and large portions of it will change over time. Beyond successful development the most important item for you as a StakeHolder is how do you get DataFace and from where.
\subsection{Get DataFace}
\end{onehalfspace}
\begin{mdframed}[roundcorner=10pt,leftmargin=1, rightmargin=1,
linecolor=green!40!brown!80,outerlinewidth=.9,
innerleftmargin=8,innertopmargin=8,innerbottommargin=8]
\begin{center}
\begin{minipage}[c]{1.0\textwidth}
To view the current DataFace project you may download the latest APK to use on your Android based device: \href{https://github.com/trentonknight/DataFaceProject/blob/master/DataFace/build/apk/DataFace-release-unsigned.apk?raw=true}{Dataface-release-unsigned.apk}
\end{minipage}
\end{center}
\end{mdframed}
\begin{mdframed}[roundcorner=10pt,leftmargin=1, rightmargin=1,
linecolor=green!40!brown!80,outerlinewidth=.9,
innerleftmargin=8,innertopmargin=8,innerbottommargin=8]
\begin{center}
\begin{minipage}[c]{1.0\textwidth}
To view the latest development source code of this product, please take a look at the online GitHub repository: \href{https://github.com/trentonknight/DataFaceProject}{DataFaceProject}
\end{minipage}
\end{center}
\end{mdframed}
\begin{mdframed}[roundcorner=10pt,leftmargin=1, rightmargin=1,
linecolor=green!40!brown!80,outerlinewidth=.9,
innerleftmargin=8,innertopmargin=8,innerbottommargin=8]
\begin{center}
\begin{minipage}[c]{1.0\textwidth}
To clone this project using git:\\ git~clone~\href{https://github.com/trentonknight/DataFaceProject.git}{https://github.com/trentonknight/DataFaceProject.git}
\end{minipage}
\end{center}
\end{mdframed}
\begin{mdframed}[roundcorner=10pt,leftmargin=1, rightmargin=1,
linecolor=green!40!brown!80,outerlinewidth=.9,
innerleftmargin=8,innertopmargin=8,innerbottommargin=8]
\begin{center}
\begin{minipage}[c]{1.0\textwidth}
To comment or see status updates on this project view the wiki here: \href{https://github.com/trentonknight/DataFaceProject/wiki/DataFace:-An-attempt-to-create-a-Object-based-learning-tool.}{DataFace: An attempt to create a Object based learning tool.}
\end{minipage}
\end{center}
\end{mdframed}
\begin{onehalfspace}
	For now DataFace is being distributed as whats referred to as Open Distribution~\citep{Open}. Once the application has reach full maturity and is releasable to the general public it will be licenced and distributed on Google Play~\citep{play}. 
\subsection{Additional Help}	
	Additional details may be found in the Handout from our previous \href{http://youtu.be/zNNrX_RKESk}{Google Hangout on Air} meeting here: \href{https://docs.google.com/file/d/0B637cz9R6ttwbC1HaFRQODVpZ1k/edit?usp=sharing}{Presentation of DataFace} as well as a paper I have written regarding its current development \href{https://docs.google.com/file/d/0B637cz9R6ttwT2RVdDNVMjBWZ00/edit?usp=sharing}{Regis University Senior Capstone Scholarship Paper}.
\end{onehalfspace}

\clearpage
\printbibliography

\end{document}
